
 

\input{header.tex}
\begin{document}

\title{Symplectic Toric Manifolds}
\author{Simon Gr\"uning}
\address[Simon Gr\"uning]{ETH Zürich, Rämistrasse 101, 8092 Zürich}
\email[Simon Gr\"uning]{\href{mailto:simongr@student.ethz.ch}{simongr@student.ethz.ch}}

\let\Psi\psi

\maketitle

\section*{(Darboux Theorem)}


\clearpage

Throughout these seminar notes we follow closely the exposition of Yannis B\"ahni\footnote{ who was of great help in preparing these notes, thank you} \cite{ybaehn} gathered through material from Lee's \cite{lee1} \cite{lee2} and Salamon's books \cite{salamon}. We take inspiration from Ana Cannas da Silva \cite{ana} and try to follow her conventions, note thus that all manifolds are smooth unless explicitly stated otherwise.


\section{Prerequisites}

\begin{remark}
%Must be even dimensional

%All manifolds are smooth unless explicitly noted otherwise as per ana's conventions

A chart $(U,\varphi)$ is called centered on $x$ when $\varphi(x) = 0$.


%precomposition definition of $F^*$
%$F^*$ diffeomorphism is linear..


\end{remark}

\begin{proposition}[Cartan's Magic Formula]
\cite{ybaehn}
Fix a manifold $M$, a vector field $X \in \mathfrak{X}(M)$, and an $\omega \in \Omega^k(M)$ for some $k \in \mathbb{N}$. Then
\begin{align*}
\mathcal{L}_X \omega = i_X(d \omega) + d(i_X \omega).
\end{align*}
\end{proposition}

\begin{theorem}[Canonical Form Theorem for Symplectic Vector Space] \label{canform} \cite{ybaehn}
Fix $(V,\omega)$ a symplectic vector space. Then $dimV = 2n$ and there exists a basis $(a_1,...,a_n,b_1, ...,b_n)$ of $V$ such that:
\begin{align*}
\omega = \sum^n_{i=1} a^i \wedge b^i
\end{align*}
where $(a^1, ... , a^n,b^1, ... ,b^n)$ denotes the dual basis of $(a_1,...,a_n,b_1, ...,b_n)$.
\end{theorem}

\begin{definition} \cite{lee1} \label{timvec}
Fix a manifold $M$, $J \subseteq \mathbb{R}$ an interval. A time-dependent vector field on $M$ is a smooth map $X: J \times M \to TM$ such that $\forall (t,x) \in J \times M: X(t,x) \in T_x M$.
\end{definition}

\begin{definition} \cite{lee1}
An integral curve of a time-dependent vector field $X$ is a curve $\gamma \in \mathcal{C}^\infty(J_0,M)$ such that $\gamma '(t) = X(t,\gamma(t))$ for all $t \in J_0$, where $J_0 \subseteq J$ is an interval by notation as in defintion \ref{timvec}.
\end{definition}

\begin{definition} \cite{lee1}
Fix a manifold $M$, $J \subseteq \mathbb{R}$ an open interval, and $X : J\times M \to TM$ a time-dependent vector field. We call a time dependent flow of X an open subset $\mathcal{D} \subseteq J \times J \times M$ paired with a map $\Psi \in \mathcal{C}^\infty(\mathcal{D},M)$ such that the following holds for $\mathcal{D}^{(t_0,x)} := \{t\in J: (t,t_0,x) \in \mathcal{D} \}$, $\Psi^{(t_0,x)}(t) := \Psi(t,t_0,x)$, $M_{t_1,t_0} := \{ x \in M:(t_1,t_0,x) \in \mathcal{D} \}$, $\Psi_{t_1,t_0}(x):= \Psi(t_1,t_0,x)$:
\begin{enumerate}
\item For any $t_0 \in J, x \in M$, $\mathcal{D}^{(t_0,x)}$ is an open interval such that $t_0 \in \mathcal{D}^{(t_0,x)}$ and $\Psi^{(t_0,x)}(t)$ is the unique maximal integral curve of $X$ with $\Psi^{(t_0,x)}(t_0) = x$.
\item $t_1 \in \mathcal{D}^{(t_0,x)} \land y = \Psi^{(t_0,x)}(t_1) \implies \mathcal{D}^{(t_1,y)} = \mathcal{D}^{(t_0,x)} \land \Psi^{(t_1,y)} = \Psi^{(t_0,x)}$
\item For any $(t_1,t_0) \in J \times J$ we have that $M_{t_1,t_0}$ is open in $M$ and $\Psi_{t_1,t_0}: M_{t_1,t_0} \to M$ is a diffeomorphism from $M_{t_1,t_0}$ onto $M_{t_0,t_1}$ with inverse $\Psi_{t_0,t_1}$
\item $x \in M_{t_1,t_0} \land \Psi_{t_1,t_0}(x) \in M_{t_2,t_1} \implies$
\begin{align*}
x \in M_{t_2,t_0} \land \Psi_{t_2,t_1} \circ \Psi_{t_1,t_0}(x) = \Psi_{t_2, t_0}(x)
\end{align*}
\end{enumerate}
\end{definition}

%\begin{definition}
%time dependent differential k-form
%\end{definition}

\begin{theorem}[Fundamental Theorem of Time-Dependent Flows] \cite{lee1} \label{flowthm}
For any time-dependent vector field $X$, there exists a time-dependend flow of $X$.
\end{theorem}

\begin{proposition}[Fisherman's Formula] \cite{lee1}
Fix a manifold $M$. If $X : J \times M \to TM$ is a time-dependent vector field with time-dependent flow $\Psi: \mathcal{D} \to M$ then for any $\omega \in \Omega^k(M)$, $(t_1,t_0,x) \in \mathcal{D}$ we have that
\begin{align*}
\frac{d}{dt} \bigg\vert_{t=t_1} \Psi^*_{t,t_0} \omega = \Psi^*_{t_1,t_0} \left( \mathcal{L}_{X_{t_1}} \omega \right)
\end{align*}
with $X_{t_1} := X(t_1, \cdot) \in \mathfrak{X}(M)$.
\end{proposition}

\begin{proposition}[Fisherman's Formula Adapted] \cite{lee1}
Fix a manifold $M$. If $X : J \times M \to TM$ is a time-dependent vector field with time-dependent flow $\Psi: \mathcal{D} \to M$, and further, $\omega : J \times M \to \Lambda^k T^* M$ is a time-dependent differential k-form, then for any $\omega \in \Omega^k(M)$, $(t_1,t_0,x) \in \mathcal{D}$ we have that
\begin{align*}
\frac{d}{dt} \bigg\vert_{t=t_1} \Psi^*_{t,t_0} \omega_t = \Psi^*_{t_1,t_0} \left( \mathcal{L}_{X_{t_1}} \omega_{t_1} + \frac{d}{dt} \bigg\vert_{t=t_1} \omega_t \right)
\end{align*}
with $X_{t_1} := X(t_1, \cdot) \in \mathfrak{X}(M)$.
\end{proposition}

\begin{proof}
For any sufficiently small $\varepsilon > 0$ (i.e. such that we remain within $J$) let
\begin{align*}
F:(t_1-\varepsilon,t_1+\varepsilon)\times(t_1-\varepsilon,t_1+\varepsilon) \to \Lambda^k T^*M
\end{align*}
be defined by
\begin{align*}
F(u,v) := \Psi^*_{u,t_0}\omega_v .
\end{align*}
We compute
\begin{align*}
\frac{d}{dt}\bigg\vert_{t=t_1} \Psi^*_{t,t_0} \omega_t &= \frac{d}{dt}\bigg\vert_{t=t_1} F(t,t) & \\
&= \frac{\partial F}{\partial u}(t_1,t_1) + \frac{\partial F}{\partial v} (t_1,t_1) &  \\
&= \frac{d}{du}\bigg\vert_{u=t_1} \Psi^*_{u,t_0} \omega_{t_1} + \frac{d}{dv} \bigg\vert_{v=t_1} \Psi^*_{t_1,t_0} \omega_v &  \\
&= \Psi^*_{t_1,t_0} (\mathcal{L}_{X_{t_1}} \omega_{t_1}) + \Psi^*_{t_1,t_0} \left( \frac{d}{dv}\bigg\vert_{v=t_1} \omega_v \right) & \text{fisherman} \\
&=  \Psi^*_{t_1,t_0} \left( \mathcal{L}_{X_{t_1}} \omega_{t_1}) +  \frac{d}{dv}\bigg\vert_{v=t_1} \omega_v \right) & 
\end{align*}
to show the required statement.
\end{proof}

\begin{lemma}\label{auxprop1} \cite{ybaehn}
Let $M$ be a manifold, $x \in M$ with basis $(e_i)$ for $T_x M$. Then there exists a chart $(U, x^1, ... ,x^n)$ centered on $x$ such that for any $i = 1, ... , n$:
\begin{align*}
\frac{\partial}{\partial x^i} \bigg\vert_x = e_i
\end{align*}
\end{lemma}

\begin{proposition} \cite{lee1}
Every smooth manifold admits a Riemannian metric.
\end{proposition}

\begin{definition}[Tubular Neighbourhood] \cite{lee2}
Let $(M,g)$  be a Riemannian manifold, $S \subseteq M$ an embedded submanifold. Denote by $\pi : NS \to S$ the normal bundle of $S$ in $M$. Restrict the exponential map of $M$ as $exp_S : \mathcal{E} \cap NS \to M$ with $\mathcal{E} \subseteq TM$. A neighbourhood $U$ of $S$ in $M$ is called a tubular neighbourhood of $S$ if there exists a positive continuous function $\delta : S \to \mathbb{R}$ such that $U$ is the diffeomorphic image under $exp_S$ of a subset $V \subseteq \mathcal{E} \cap NS$ of the form 
\begin{align*}
V = \{ (x,v) \in NS : |v|_g < \delta(x) \} .
\end{align*}
We call $U$ a uniform tubular neighbourhood of $S$ if $\delta$ is constant.
\end{definition}

\begin{theorem}[Existence of Tubular N] \label{extub} \cite{lee2}
For every embedded submanifold of a Riemannian manifold $(M,g)$, there exists a tubular neighbourhood in $M$. If the submanifold is compact, there exists a uniform tubular neighbourhood.
\end{theorem}

\begin{proposition}[Homotopy Formula] \label{homoform} \cite{ana}
Fix $U$, a tubular neighburhood of a submanifold $S$ embedded in $M$. If $\omega \in \Omega^k(U)$ is closed and $\iota^* \omega = 0$ for some $\iota : S \hookrightarrow U$, then there exists an $\eta \in \Omega^{k-1}(U)$ with $\omega = d\eta$. It is possible to ensure that $\forall x \in S: \eta_x = 0$.
\end{proposition}

\begin{proof}
By definition of tubular neighbourhood, we have a positive continous function $\delta : S \to \mathbb{R}$ with 
\begin{align*}
U = exp_S(\{ (x,v) \in NS : |v|_g < \delta(x) \})
\end{align*}
Fix a $t \in I =[0,1]$. Let $\Psi_t: U \to U$ be defined by
\begin{align*}
\Psi_t(exp_S(x,v)) := exp_S(x,tv) .
\end{align*}
Since $exp_S$ is injective and we have an smooth inverse $exp_S(x,tv) \mapsto exp_S(x,v)$, $\Psi_t$ is a diffeomorphism for $t>0$.
The proof is complete if we find a map $H: \Omega^k(U) \to \Omega^{k-1}(U)$  with
\begin{align*}
H \circ d + d \circ H = \Psi^*_1 - \Psi^*_0 = id - \iota^*
\end{align*}
since it then follows by the assumptions on $\omega$ that
\begin{align*}
H  d(\omega) + d  H(\omega) &= id(\omega) - \iota^*(\omega) \\
d(H\omega) &= \omega .
\end{align*} 
\underline{Claim:} We can define such a map as:
\begin{align*}
(H(\omega))_x(v) &:= \int^1_0 (\Psi^*_t(i_{X_t} \omega))_x(v) dt \\
&= \int^1_0 \omega_{\Psi_t(x)} \left( \frac{d}{dt} \Psi_t(x), D(\Psi_t)_x(v) \right) dt
\end{align*}
for $x\in U , v\in T_xU$, and $X_t \in \mathfrak{X}(U)$ given for $t>0$ by
\begin{align*}
X_t := \left( \frac{d}{dt} \Psi_t \right) \circ \Psi^{-1}_t .
\end{align*}
\underline{Proof of Claim:}
We compute
\begin{align*}
H(d\omega) + d(H\omega) &= \int^1_0 \Psi^*_t(i_{X_t}(d\omega))+ d(\Psi^*_t(i_{X_t}\omega)) dt & \\
&= \int^1_0 \Psi^*_t(i_{X_t}(d\omega) + d i_{X_t}\omega) dt &  \\
&= \int^1_0 \Psi^*_t (\mathcal{L}_{X_t} \omega) dt & \text{cartan} \\
&= \int^1_0 \frac{d}{dt} \Psi^*_t \omega dt & \text{fisherman}\\
&= \Psi^*_1 \omega - \Psi^*_0 \omega 
\end{align*}
Since we have $\Psi_t|_S = id_S$ for $t \in I$, with $S$ seen as a subset of $NS$ via the zero section, it follows that $X_t$ vanishes on $S$ and so will $\eta$.
\end{proof}

%\begin{remark}
%(!) this is calles homotopy formula since bla bla... de rahm 
%\end{remark}

\begin{proposition}[Existence of Vector Field] \label{propisom} \cite{ybaehn}
Fix $(M,\omega)$ a symplectic manifold and $\eta \in \Omega^1(M)$. Then there exists a unique vector field $X \in \mathfrak{X}(M)$ such that $i_X \omega = \eta$.
\end{proposition}

\section{Moser Trick}

\begin{remark}
See \url{https://en.wikipedia.org/wiki/J%C3%BCrgen_Moser} for history.
\end{remark}

\begin{theorem}[Moser Trick]\label{mostr} \cite{salamon}
Fix $M$ a compact manifold. Suppose for some open interval $0 \in J \subseteq \mathbb{R}$ we have a smooth family of symplectic forms $(\omega_t)_{t \in J} \in \Omega^2(M)$ such that there exists another smooth family $(\eta_t)_{t \ in K} \in \Omega^1(M)$ with
\begin{align*}
\frac{d}{dt} \omega_t = d\eta_t.
\end{align*}
Then there exists a family of diffeomorphisms $(\Psi_t)_{t \in J} \in Diff(M)$ with
\begin{align*}
\Psi^*_t \omega_t = \omega_0.
\end{align*}
\end{theorem}

\begin{proof}
The Moser trick is to see the $\Psi_t$ as time-dependent flows induced by some $X_t$ time-dependent vector fields. Note that here we use the convention $\Psi_t := \Psi_{t,0}$ where $t_0$ is fixed as $0$.

To begin with the end in mind, suppose that
\begin{align*}
\frac{d}{dt} \Psi_t = X_t \circ \Psi_t
\end{align*}

If we would have the $X_t$ and were to induce the flow $\Psi_t$ with the Fundamental Flow theorem we would also receive that
\begin{align*}
\Psi_0 = \Psi_0 \circ \Psi_0 = id_M
\end{align*}

To satisfy $\Psi^*_t \omega_t$ being constant as desired we set:
\begin{align*}
0 = \frac{d}{dt} \Psi^*_t \omega_t &= \Psi^*_t \left( \mathcal{L}_{X_t} \omega_t + \frac{d}{dt} \omega_t \right) &\text{fisherman}\\
&= \Psi^*_t \left( i_{X_t}(d\omega_t) + d(i_{X_t}\omega_t)+\frac{d}{dt}\omega_t \right) & \text{cartan} \\
&= \Psi^*_t \left( d(i_{X_t}\omega_t)+\frac{d}{dt}\omega_t \right) & d\omega_t=0 \text{ since closed}\\
&= \Psi^*_t \left( d(i_{X_t}\omega_t)+d\eta_t  \right) & \text{assumption} \\
\end{align*}
Since $\Psi^*_t$ is an isomorphism and $d$ is a sheaf morphism we can peel away the layers:
\begin{align*}
0 &= \Psi^*_t \left( d(i_{X_t}\omega_t)+d\eta_t  \right) \\
\Leftrightarrow 0 &= d(i_{X_t}\omega_t)+d\eta_t = d(i_{X_t}\omega_t +\eta_t )\\
\Leftrightarrow 0 &= i_{X_t}\omega_t +\eta_t
\end{align*}

We can solve $i_{X_t}\omega_t = -\eta_t$ for $X_t$ explicitly with $X_t = -\Omega^{-1}_t(\eta_t)$ where $\Omega_t$ is the tangent-cotangent bundle isomorphism.  With the Fundamental theorem of time-dependent Flow we can now integrate the $X_t$ resulting in the flows $\Psi_t$ such that $\Psi^*_t \omega_t$ is constant, and since $\Psi^*_0 = id$ we have $\Psi^*_t \omega_t = \omega_0$. 

%(!) note smoothly from t -> allows applic of flo thm.
\end{proof}



\begin{theorem}[Moser Isotopy]\label{mosiso} \cite{salamon}
Fix as $M$ a 2n-dimensional manifold and as $S \subseteq M$ a compact submanifold. If $\omega_0 , \omega_1 \in \Omega^2(M)$ are close and
\begin{enumerate}
\item $\forall x \in S: \omega_0|x = \omega_1|x$
\item $\forall x \in S: \omega_0|x, \omega_1|x$ are nondegenerate.
\end{enumerate}
then there exist neighbourhoods $U_0,U_1$ of $S$ in $M$ and a diffeomorphism $F: U_0 \to U_1$ with
\begin{align*}
F|_S &= id_S \\
F^*(\omega_1|_{U_1}) &= \omega_0|_{U_0}.
\end{align*}
\end{theorem}

\begin{proof}
Let $U$ be a uniform tubular neighbourhood of $S$ in $M$ by the Theorem on the Existence of Tubular Neighbourhoods. By construction $\overline{U}$ is compact, hence the uniform nature. By the Homotopy Formula \ref{homoform} there exists $\eta \in \Omega^{1}(U)$ such that
\begin{align*}
\omega_1 - \omega_0 = d\eta.
\end{align*}
Further we can ensure in the application of the Homotopy Formula that $\eta$ vanishes on $S$.
Define for $t \in \mathbb{R}$
\begin{align*}
\omega_t := \omega_0 + t(\omega_1 - \omega_0)
\end{align*}
By construction $\omega_t$ is closed. To assure that $\omega_t$ is non-degenerate, we shrink $U$ to $U_0$, a new neighbourhood of $S$ in $M$. In doing this, note that $\omega_t = \omega_0$ on $S$ per assumption and that we may take the union of open neighbourhoods of the non-degenerate points of $S$ to exceed $S$ as it is closed, and by smoothness retain the non-degenerate property.

We then have that:
\begin{align*}
\frac{d}{dt} \omega_t = \omega_1 - \omega_0 = d\eta .
\end{align*}
Since $\overline{U_0} \subseteq \overline{U}$ is compact due to being a closed subset of a compact space, we can apply now Moser's Trick \ref{mostr} to get a family of diffeomorphisms $(\Psi_t)_{t \in J}$ with
\begin{align*}
\Psi^*_t \omega_t = \omega_0.
\end{align*}
Let now $F := \Psi_1$, $U_1 := F(U_0)$. The final property follows from $\eta$ vanishing on $S$.
\end{proof}

\section{Darboux Theorem}

\begin{theorem}[Darboux's Theorem] \cite{lee1}
Fix $(M , \omega)$ a 2n-dimensional symplectic manifold, $x \in M$. Then there exists a chart $(U, x^1, ... , x^n, y^1, .... , y^n)$ centered on $x$ such that:
\begin{align*}
\omega |_U = \sum^n_{i=1} dx^i \wedge dy^i
\end{align*}.
\end{theorem}

\begin{proof}
The canonical form theorem for symplectic tensors \ref{canform} provides us a basis $(a_1, ... , a_n, b_1, ... , b_n)$ for $T_x M$ such that for its dual basis $(a^1, ... , a^n, b^1, ... , b^n)$ we have
\begin{align*}
\omega_x = \sum^n_{i=1} da^i \wedge db^i.
\end{align*}

By proposition \ref{auxprop1} we further have a chart $(U, \tilde{\varphi})$ centered on $x$ with associated coordinates $({\tilde{x}}^1, ... , {\tilde{x}}^n ,{\tilde{y}}^1, ... , {\tilde{y}}^n)$ such that for $i = 1,..., n$
\begin{align*}
\frac{\partial}{\partial \tilde{x}^i} \bigg\vert_x = a_i  \\
\frac{\partial}{\partial \tilde{y}^i} \bigg\vert_x = b_i 
\end{align*}

Combining the previous two results and traversing again into the dual basis we have
\begin{align*}
\omega_x = \sum^n_{i=1} d\tilde{x}^i|_x \wedge d\tilde{y}^i|_x.
\end{align*}

Define:

\begin{align*}
\omega_0 := \omega|_U \\
\omega_1 := \sum^n_{i=1} d\tilde{x}^i \wedge d\tilde{y}^i.
\end{align*}

Then $\omega_0 , \omega_1$ are symplectic forms on $U$. 

Application of the Moser isotopy \ref{mosiso} to the compact submanifold $\{x\} \subseteq U$ given $\omega_0, \omega_1$ provides the existence of neighbourhoods $U_0, U_1$ of $\{x\}$ in $U$ and a diffeomorphism $F: U_0 \to U_1$ with
\begin{align*}
F(x) = x \\
F^* \omega_1 = \omega_0.
\end{align*}

Define now another chart $(U_0,\varphi)$ with $\varphi := \tilde{\varphi}|_{U_1} \circ  F$. By construction the associated coordinates are
\begin{align*}
x^i = \tilde{x}^i \circ F \\
y^i = \tilde{y}^i \circ F.
\end{align*}

If then follows that $\varphi(x) = \tilde{\varphi}(x) = 0$, or in other words, the chart remains centered. The remaining property of our chart $(U_0,\varphi)$ follows by:

\begin{align*}
\omega|_{U_0} &= \omega_0|_{U_0} \\
&= F^* (\omega_1|{U_1}) \\
&= F^* \left( \sum^n_{i=1} d\tilde{x}^i \wedge d\tilde{y}^i \right) \\
&= \sum^n_{i=1} F^* (d\tilde{x}^i \wedge d\tilde{y}^i) \\
&= \sum^n_{i=1} F^*(d\tilde{x}^i) \wedge F^*(d\tilde{y}^i) \\
&= \sum^n_{i=1} d F^*(\tilde{x}^i) \wedge d F^*(\tilde{y}^i) \\
&= \sum^n_{i=1} d (\tilde{x}^i \circ F) \wedge d ( \tilde{y}^i \circ F) \\
&= \sum^n_{i=1} dx^i \wedge dy^i.
\end{align*}






\end{proof}

\begin{remark}
This theorem allows us to examine the local nature of symplectic manifolds and discover that they are always similar in nature in this regard.
\end{remark}

%\section{Discussion / Applications}

%bla

\clearpage

\begin{appendix}

\section{Auxiliary}

%\begin{enumerate}
%\item symplectic manifold
%\item symplectic form
%\item riemanian manifold
%\item embedded submanifold
%\item smooth manifold
%\item einstein summation convention
%\item $T_x M$ et al.
%\item basis of above and dual basis
%\item coordinates associated to chart?
%\item time dependent vector field and flow
%\item time dep differential k-form
%\item interior multiplication $i_X$.
%\item exp..
%\item NS
%\item Diff(M)
%\item sheaf morphism
%\item $|\cdot |_g$ riemannian metric
%\item pullback
%\item symplectic vector space
%\item $\iota$ inclusion
%\item derivative D
%\end{enumerate}


\begin{lemma}\cite{ybaehn}
For a smooth function $F$ from manifolds $M$ to $N$ and $\omega,\eta \in \Omega(N)$ we have
\begin{align*}
F^*(\omega \wedge \eta) = F^* \omega \wedge F^* \eta
\end{align*}
\end{lemma}

\begin{lemma}\cite{ybaehn}
For a smooth function $F$ from manifolds $M$ to $N$ and $\omega \in \Omega(M)$ we have
\begin{align*}
F^*(d\omega) = d(F^* \omega).
\end{align*}
\end{lemma}


\end{appendix}

%\clearpage
\vspace{8cm}

%\bibliographystyle{IEEEabrv,publishers,confs-jrnls,biblio} %IEEEtran
%\bibliography{IEEEabrv,publishers,confs-jrnls,biblio} % TODO
%\bibliography{biblio}
%\printbibliography
\printbibliography

\end{document}