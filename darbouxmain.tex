\input{header.tex}
\begin{document}

\title{Toric Symplectic Manifolds}
\author{Simon Gr\"uning}
\address[Simon Gr\"uning]{University of Zurich, R\"{a}mistrasse 71, 8006 Zurich}
\email[Simon Gr\"uning]{\href{mailto:simon.gruening@uzh.ch}{simon.gruening@uzh.ch}}


\maketitle

\section*{(Darboux Theorem)}


\clearpage

(!) Throghout the seminar notes we follow closely the exposition of Yannis gathered of material from Lee and Salomon's books. We take inspiration from Anna's stuff.



\section{Prerequisites}

\begin{remark}
Must be even dimensional

All manifolds are smooth unless noted otherwise as per ana's conventions

chart centered on x means $\varphi(x) = 0$.


precomposition definition of $F^*$
$F^*$ diffeomorphism is linear..


\end{remark}

\begin{proposition}[Cartan's Magic Formula]
(! ref lee)

Fix a manifold $M$, a vector field $X \in \mathfrak{X}(M)$, and an $\omega \in \Omega^k(M)$ for some $k \in \mathbb{N}$. Then
\begin{align*}
\mathcal{L}_X \omega = i_X(d \omega) + d(i_X \omega).
\end{align*}
\end{proposition}

\begin{theorem}\label{canform}
Canonical Form theorem
\end{theorem}

\begin{definition}
time dependent vector field
\end{definition}

\begin{definition}
time dependent flow of X
\end{definition}

\begin{definition}
time dependent differential k-form
\end{definition}

\begin{proposition}[Fisherman's Formula]
(!ref lee)
Fix a manifold $M$. If $X : J \times M \to TM$ is a time-dependent vector field with time-dependent flow $\Psi: \mathcal{D} \to M$ then for any $\omega \in \Omega^k(M)$, $(t_1,t_0,x) \in \mathcal{D}$ we have that
\begin{align*}
\frac{d}{dt} \bigg\vert_{t=t_1} \Psi^*_{t,t_0} \omega = \Psi^*_{t_1,t_0} \left( \mathcal{L}_{X_{t_1}} \omega \right)
\end{align*}
with $X_{t_1} := X(t_1, \cdot) \in \mathfrak{X}(M)$.
\end{proposition}

\begin{proposition}[Fisherman's Formula Adapted]
Fix a manifold $M$. If $X : J \times M \to TM$ is a time-dependent vector field with time-dependent flow $\Psi: \mathcal{D} \to M$, and further, $\omega : J \times M \to \Lambda^k T^* M$ is a time-dependent differential k-form, then for any $\omega \in \Omega^k(M)$, $(t_1,t_0,x) \in \mathcal{D}$ we have that
\begin{align*}
\frac{d}{dt} \bigg\vert_{t=t_1} \Psi^*_{t,t_0} \omega_t = \Psi^*_{t_1,t_0} \left( \mathcal{L}_{X_{t_1}} \omega_{t_1} + \frac{d}{dt} \bigg\vert_{t=t_1} \omega_t \right)
\end{align*}
with $X_{t_1} := X(t_1, \cdot) \in \mathfrak{X}(M)$.
\end{proposition}

\begin{proof}
todo chain rule amazing
\end{proof}

\begin{lemma}\label{auxprop1}
Let $M$ be a manifold, $x \in M$ with basis $(e_i)$ for $T_x M$. Then there exists a chart $(U, x^1, ... ,x^n)$ centered on $x$ such that for any $i = 1, ... , n$:
\begin{align*}
\frac{\partial}{\partial x^i} \bigg\vert_x = e_i
\end{align*}
\end{lemma}

\begin{definition}[Tubular Neighbourhood]
(! ref lee)

\end{definition}

\begin{theorem}[Existence of Tubular N] \label{extub}
ahhh
\end{theorem}

\begin{proposition}[Homotopy Formula] \label{homoform} (!ref canas).
Fix $U$, a tubular neighburhood of a submanifold $S$ embedded in $M$. If $\omega \in \Omega^k(U)$ is closed and $i^* \omega = 0$ for some $i : S \hookrightarrow U$, then there exists an $\eta \in \Omega^{k-1}(U)$ with $\omega = d\eta$ and $\forall x \in S: \eta_x = 0$.
\end{proposition}

\begin{proof}
By definition of tubular neighbourhood, we have a continous function $\delta : S \to \mathbb{R}$ with 
\begin{align*}
U = exp_S(\{ (x,v) \in NS : |v|_g < \delta(x) \})
\end{align*}

..... todo

\end{proof}

\begin{proposition}[Existence of Vector Field] \label{propisom}
Fix $(M,\omega)$ a symplectic manifold and $\eta \in \Omega^1(M)$. Then there exists a unique vector field $X \in \mathfrak{X}(M)$ such that $i_X \omega = \eta$.
\end{proposition}

\section{Moser Trick}

\begin{remark}
Moser was at ETH etc. (!)
This trick is very useful..
\end{remark}

\begin{theorem}[Moser Trick]\label{mostr}
(! ref salamon)
Fix $M$ a compact manifold. Suppose for some open interval $0 \in J \subseteq \mathbb{R}$ we have a smooth family (!) of symplectic forms $(\omega_t)_{t \in J} \in Omega^2(M)$ such that there exists another smooth family $(\eta_t)_{t \ in K} \in \Omega^1(M)$ with
\begin{align*}
\frac{d}{dt} \omega_t = d\eta_t.
\end{align*}
Then there exists a family of diffeomorphisms $(\Psi_t)_{t \in J} \in Diff(M)$ with
\begin{align*}
\Psi^*_t \omega_t = \omega_0.
\end{align*}
\end{theorem}

\begin{proof}
bla bla bla (!)

note: $\Psi_t := \Psi_{t,0}$ with t0 set as 0 (can translate etc.)

To begin with the end in mind, suppose that
\begin{align*}
\frac{d}{dt} \Psi_t = X_t \circ \Psi_t
\end{align*}

If we would have the $X_t$ and were to induce the flow $\Psi_t$ with the Fundamental Flow theorem (!) we would also receive that
\begin{align*}
\Psi_0 = \Psi_0 \circ \Psi_0 = id_M
\end{align*}

To satisfy $\Psi^*_t \omega_t$ being constant as desired we set:
\begin{align*}
0 = \frac{d}{dt} \Psi^*_t \omega_t &= \Psi^*_t \left( \mathcal{L}_{X_t} \omega_t + \frac{d}{dt} \omega_t \right) &fisherman's formula \\
&= \Psi^*_t \left( i_{X_t}(d\omega_t) + d(i_{X_t}\omega_t)+\frac{d}{dt}\omega_t \right) & cartans \\
&= \Psi^*_t \left( d(i_{X_t}\omega_t)+\frac{d}{dt}\omega_t \right) & d\omega_t=0 b/c w_t closed \\
&= \Psi^*_t \left( d(i_{X_t}\omega_t)+d\eta_t  \right) & assumption \\
\end{align*}
Since $\Psi^*_t$ is an isomorphism and $d$ is a sheaf morphism we can peel away the layers:
\begin{align*}
0 &= \Psi^*_t \left( d(i_{X_t}\omega_t)+d\eta_t  \right) \\
\Leftrightarrow 0 &= d(i_{X_t}\omega_t)+d\eta_t = d(i_{X_t}\omega_t +\eta_t )\\
\Leftrightarrow 0 &= i_{X_t}\omega_t +\eta_t
\end{align*}

We can solve $i_{X_t}\omega_t = -\eta_t$ for $X_t$ using Proposition \ref{propisom}.  With the Flow Theorem (!) we can now integrate the $X_t$ resulting in the flows $\Psi_t$ such that $\Psi^*_t \omega_t$ is constant, and since $\Psi^*_0 = id$ we have $\Psi^*_t \omega_t = \omega_0$. 

(!) note smoothly from t -> allows applic of flo thm.
\end{proof}



\begin{theorem}[Moser Isotopy]\label{mosiso}
(!salomon)
Fix as $M$ a 2n-dimensional manifold and as $S \subseteq M$ a compact submanifold. If $\omega_0 , \omega_1 \in \Omega^2(M)$ are close and
\begin{enumerate}
\item $\forall x \in S: \omega_0|x = \omega_1|x$
\item $\forall x \in S: \omega_0|x, \omega_1|x$ are nondegenerate.
\end{enumerate}
then there exist neighbourhoods $U_0,U_1$ of $S$ in $M$ and a diffeomorphism $F: U_0 \to U_1$ with
\begin{align*}
F|_S &= id_S \\
F^*(\omega_1|_{U_1}) &= \omega_0|_{U_0}.
\end{align*}
\end{theorem}

\begin{proof}
Let $U$ be a uniform tubular neighbourhood of $S$ in $M$ by Theorem \ref{extub}. By construction $\overline{U}$ is compact. By the Homotopy Formula \ref{homoform} there exists $\eta \in \Omega^{1}(U)$ such that
\begin{align*}
\omega_1 - \omega_0 = d\eta.
\end{align*}
Further we also have that $\eta$ vanishes on $S$.
Define for $t \in \mathbb{R}$
\begin{align*}
\omega_t := \omega_0 + t(\omega_1 - \omega_0)
\end{align*}
By construction $\omega_t$ is closed (!). To assure that $\omega_t$ is non-degenerate, we shrink $U$ to $U_0$, a new neighbourhood of $S$ in $M$. In doing this, note that $\omega_t = \omega_0$ on $S$ per assumption and that we may take the union of open neighbourhoods of the non-degenerate points of $S$ to exceed $S$ as it is closed, and by smoothness retain the non-degenerate property.

We then have that:
\begin{align*}
\frac{d}{dt} \omega_t = \omega_1 - \omega_0 = d\eta .
\end{align*}
Since $\overline{U_0} \subseteq \overline{U}$ is compact due to being a closed subset of a compact space, we can apply now Moser's Trick \ref{mostr} to get a family of diffeomorphisms $(\Psi_t)_{t \in J}$ with
\begin{align*}
\Psi^*_t \omega_t = \omega_0.
\end{align*}
Let now $F := \Psi_1$, $U_1 := F(U_0)$. The final property follows from $\eta$ vanishing on $S$ (!).
\end{proof}

\section{Darboux Theorem}

\begin{theorem}[Darboux's Theorem]
Fix $(M , \omega)$ a 2n-dimensional symplectic manifold, $x \in M$. Then there exists a chart $(U, x^1, ... , x^n, y^1, .... , y^n)$ centered on $x$ such that:
\begin{align*}
\omega |_U = \sum^n_{i=1} dx^i \wedge dy^i
\end{align*}.
\end{theorem}

\begin{proof}
The canonical form theorem for symplectic tensors \ref{canform} provides us a basis $(a_1, ... , a_n, b_1, ... , b_n)$ for $T_x M$ such that for its dual basis $(a^1, ... , a^n, b^1, ... , b^n)$ we have
\begin{align*}
\omega_x = \sum^n_{i=1} da^i \wedge db^i.
\end{align*}

By proposition \ref{auxprop1} we further have a chart $(U, \tilde{\varphi})$ centered on $x$ with associated coordinates $({\tilde{x}}^1, ... , {\tilde{x}}^n ,{\tilde{y}}^1, ... , {\tilde{y}}^n)$ such that for $i = 1,..., n$
\begin{align*}
\frac{\partial}{\partial \tilde{x}^i} \bigg\vert_x = a_i  \\
\frac{\partial}{\partial \tilde{y}^i} \bigg\vert_x = b_i 
\end{align*}

Combining the previous two results and traversing again into the dual basis we have
\begin{align*}
\omega_x = \sum^n_{i=1} d\tilde{x}^i|_x \wedge d\tilde{y}^i|_x.
\end{align*}

Define:

\begin{align*}
\omega_0 := \omega|_U \\
\omega_1 := \sum^n_{i=1} d\tilde{x}^i \wedge d\tilde{y}^i.
\end{align*}

Then $\omega_0 , \omega_1$ are symplectic forms on $U$ (!). 

Application of the Moser isotopy \ref{mosiso} to the compact submanifold ${x} \subseteq U$ given $\omega_0, \omega_1$ provides the existence of neighbourhoods $U_0, U_1$ of $x$ in $U$ and a diffeomorphism $F: U_0 \to U_1$ with
\begin{align*}
F(x) = x \\
F^* \omega_1 = \omega_0.
\end{align*}

Define now another chart $(U_0,\varphi)$ with $\varphi := \tilde{\varphi}|_{U_1} \circ  F$. By construction (!) the associated coordinates are
\begin{align*}
x^i = \tilde{x}^i \circ F \\
y^i = \tilde{y}^i \circ F.
\end{align*}

If then follows (!) that $\varphi(x) = \tilde{\varphi}(x) = 0$. The remaining property of our chart $(U_0,\varphi)$ follows by:

\begin{align*}
\omega|_{U_0} &= \omega_0|_{U_0} \\
&= F^* (\omega_1|{U_1}) \\
&= F^* \left( \sum^n_{i=1} d\tilde{x}^i \wedge d\tilde{y}^i \right) \\
&= \sum^n_{i=1} F^* (d\tilde{x}^i \wedge d\tilde{y}^i) \\
&= \sum^n_{i=1} F^*(d\tilde{x}^i) \wedge F^*(d\tilde{y}^i) \\
&= \sum^n_{i=1} d F^*(\tilde{x}^i) \wedge d F^*(\tilde{y}^i) \\
&= \sum^n_{i=1} d (\tilde{x}^i \circ F) \wedge d ( \tilde{y}^i \circ F) \\
&= \sum^n_{i=1} dx^i \wedge dy^i.
\end{align*}






\end{proof}

\section{Discussion / Applications}

bla

\begin{appendix}

\section{Basics}

\begin{enumerate}
\item symplectic manifold
\item symplectic form
\item smooth manifold
\item einstein summation convention
\item $T_x M$ et al.
\item basis of above and dual basis
\item coordinates associated to chart?
\item time dependent vector field and flow
\item time dep differential k-form
\item interior multiplication $i_X$.
\item exp..
\item NS
\item Diff(M)
\item sheaf morphism
\end{enumerate}


\begin{lemma}[refyan E.]
For a smooth function $F$ from manifolds $M$ to $N$ and $\omega,\eta \in \Omega(N)$ we have
\begin{align*}
F^*(\omega \wedge \eta) = F^* \omega \wedge F^* \eta
\end{align*}
\end{lemma}

\begin{lemma}[refyan E.203]
For a smooth function $F$ from manifolds $M$ to $N$ and $\omega \in \Omega(M)$ we have
\begin{align*}
F^*(d\omega) = d(F^* \omega).
\end{align*}
\end{lemma}


\end{appendix}

\end{document}